\documentclass{article}
\usepackage[utf8]{inputenc}

\title{Analysis of High-Energy Particles in Quantum Vacuum}
\author{Dr. Sergio Lazaro}

\begin{document}

\maketitle

\section{Abstract}
This paper explores the distribution of quantum fluctuations in non-inertial frames. We utilize the Optimal Transport theory to measure the divergence between vacuum states. This version includes updated results.

\section{Introduction}
Current models of quantum field theory rely on the Bogoliubov transformation to describe particle creation. However, the computational cost of comparing these states is often prohibited for large N systems.

\section{Methodology}
We define the energy density distribution $\rho(x)$ as a set of discrete points in a 3D grid.
The OTX algorithm provides a novel way to compute Wasserstein distances in O(N log N) time, making REAL-TIME analysis possible.
The distance between two states $A$ and $B$ is given by:

\begin{equation}
W_p(A, B) = \left( \inf_{\gamma \in \Gamma(A, B)} \int_{X \times Y} d(x, y)^p d\gamma(x, y) \right)^{1/p} + \alpha
\end{equation}

Wait! We added a factor $\alpha$ to the equation to account for dark energy.

\section{Results}
The results show a clear alignment between predicted vacuum decay and experimental observations. The Wasserstein metric proves more sensitive than Kullback-Leibler divergence for these topological changes.

\section{Conclusion}
OTX represents a significant leap in computational physics, specifically for distribution matching in high-dimensional manifolds. Future research will explore integration with GPU clusters.

\end{document}
